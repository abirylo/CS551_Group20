\documentclass{article}
\title{CS 551 Project 1 Report}
\author{Group 20}
\date{September 14, 2013}

\usepackage{fullpage}
\usepackage{enumerate}

\usepackage{fancyvrb}
\DefineVerbatimEnvironment{code}{Verbatim}{fontsize=\small}
\DefineVerbatimEnvironment{example}{Verbatim}{fontsize=\small}
\newcommand{\ignore}[1]{}

\usepackage[usenames]{color}
\usepackage{listings}
\lstset{ %
	language=bash,                			% choose the language of the code
	basicstyle=\footnotesize,       			% the size of the fonts that are used for the code
	numbers=left,                   			% where to put the line-numbers
	numberstyle=\footnotesize,				% the size of the fonts that are used for the line-numbers
	stepnumber=1,                   			% the step between two line-numbers. If it is 1 each line will be numbered
	numbersep=5pt,                  			% how far the line-numbers are from the code
	backgroundcolor=\color{white},			% choose the background color. You must add \usepackage{color}
	showspaces=false,               			% show spaces adding particular underscores
	showstringspaces=false,         			% underline spaces within strings
	showtabs=false,                 			% show tabs within strings adding particular underscores
	frame=single,                   			% adds a frame around the code
	tabsize=2,              					% sets default tabsize to 2 spaces
	captionpos=t,	                   		% sets the caption-position to bottom
	breaklines=true,        					% sets automatic line breaking
	breakatwhitespace=false,    				% sets if automatic breaks should only happen at whitespace
	escapeinside={\%}{)}          			% if you want to add a comment within your code
}

\begin{document}
\maketitle

\section{Basic Operation}

\begin{lstlisting}
# ./shell
Oldenv:/root/bin:/usr/local/bin:/bin:/sbin:/usr/bin:/usr/sbin:/usr/pkg/bin:/usr/pkg/sbin:/usr/pkg/X11R6/bin 
Newenv: /root:/usr/bin:/bin: 
$
$pwd
/root
$
$cd project1
$
$pwd
/root/project1
$
$ls -alh
total 146K
drwxr-xr-x  2 root  operator  576B Sep 14 21:17 .
drwxr-xr-x  4 root  operator  768B Sep 14 19:36 ..
-rw-r--r--  1 root  operator  297B Sep 14 19:36 Makefile
-rw-r--r--  1 root  operator   70B Sep 14 19:37 profile.txt
-rwxr-xr-x  1 root  operator   60K Sep 14 20:08 shell
-rw-r--r--  1 root  operator   24K Sep 14 20:08 shell.c
-rw-r--r--  1 root  operator   60K Sep 14 20:08 shell.o
$
$cat profile.txt
PROMPT=$
HOME_DIRECTORY=/root
ALARM=ON
ALARM_TIME=5
MAX_ARGUMENTS=128
$
$ls -l > output.txt
$
$ls
Makefile	Report		output.txt	profile.txt	shell		shell.c		shell.o
$cat output.txt
total 146K
drwxr-xr-x  2 root  operator  576B Sep 14 21:17 .
drwxr-xr-x  4 root  operator  768B Sep 14 19:36 ..
-rw-r--r--  1 root  operator  297B Sep 14 19:36 Makefile
-rwxr-xr-x  1 root  operator    0B Sep 14 21:17 output.txt
-rw-r--r--  1 root  operator   70B Sep 14 19:37 profile.txt
-rwxr-xr-x  1 root  operator   60K Sep 14 20:08 shell
-rw-r--r--  1 root  operator   24K Sep 14 20:08 shell.c
-rw-r--r--  1 root  operator   60K Sep 14 20:08 shell.o
$
$cat output.txt | wc
       10      83     511
$
$wc output.txt
       10      83     511 output.txt
$
$sleep 10
Child process (pid: 15098) has been running longer than 5 seconds, terminate? (y/n): y
Pid 15098 killed.
$
$sleep 10
Child process (pid: 15100) has been running longer than 5 seconds, terminate? (y/n): n
$
$sleep 10 | sleep 10
Child processes (pids: 15101, 15102) have been running longer than 5 seconds, terminate them? (y/n): y
Pid 15101 killed.
Pid 15102 killed.
$
$if 10 -eq 10 then echo "true" else echo "false" fi
"true"
$
$quit
ERROR: command: quit FAILED
$
$exit
#
\end{lstlisting}

\end{document}